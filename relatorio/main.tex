\documentclass[a4paper]{article}
\usepackage[portuges]{babel}
\usepackage[latin1]{inputenc}
\usepackage[T1]{fontenc}
\usepackage{fancyvrb}
\usepackage{url}

\usepackage{aeguill}  % usefull for pdflatex
%\usepackage[compat2,twosideshift=0mm,left=20mm,right=20mm,bottom=20mm,top=15mm]{geometry}

\usepackage{algorithmic}
\usepackage{algorithm}
\usepackage{float}
\usepackage{verbatim}


\parindent=2em


\title{MDIO}
\author{Gabriel Po�a \and Pedro Nunes \and Sofia Vieira}

\date{\today}

\begin{document}

\maketitle

\begin{abstract}

\end{abstract}

\parskip=0mm
%\tableofcontents
\parskip=2mm



\section{Exercicio 2} { 

	$\sum_{ i=0 }^n\sum_{ j=0 }^n\sum_{ k=0 }^m Lj * x_{ijk}$ \\ \\
	\textit{s.a:} \\ \\
	\begin{enumerate}
	$\sum_{ i=0 }^n\sum_{ k=1 }^m x_{ijk} <= 1 				\hfill j=0,\ldots,n$ \\ \\
	$\sum_{ i=0 }^nXipk  = \sum_{ j=0 }^n x_{pjk} 			\hfill p=0,\ldots,n \qquad k=1,\ldots,m$ \\ \\
	$\sum_{ i=0 }^n\sum_{ j=0 }^n t_{ij} * x_{ijk} <= Tmax 	\hfill k=1,\ldots,m$ \\ \\
	$\sum_{ i=1 }^{n-1} x_{ 0ik } = 1							\hfill k=1,\ldots,m$ \\ \\
	$\sum_{ i=1 }^{n-1} x_{ ink } = 1							\hfill k=1,\ldots,m$ \\ \\
	\end{enumerate}
	
	\begin{verbatim}
/* Fun��o Objectivo */
max: 10 e2 + 5 e3 + 5 e4 + 10 e5 + 15 e6 + 15 e7 + 15 e8 + 10 e9; 
	\end{verbatim}
}

\section{Exercicio 3}
O pseudo c�digo na Figura~\ref{pseudocodigo} representa uma possiv�l solu��o para um problema TOP.

\begin{algorithm}                     
\floatname{algorithm}{Algoritmo}
\caption{Pseudo c�digo para o exerc�cio 3.}        
\label{pseudocodigo}    
\begin{algorithmic}
\STATE $graph =$ Array com todos os v�rtices. 
\STATE $dMax =$ Valor da dist�ncia m�xima que cada veiculo pode percorrer.
\STATE $paths\gets$ Caminhos mais curtos a partir de $graph$ e $dMax$.
\WHILE{Existir vertices n�o visitados por analisar}
	\STATE $bestV \gets$ V�rtice de maior lucro n�o visitado.	
	\FOR{$path : paths$ (para todos os percursos)}	
		\FOR{$v : path$ (para todos os v�rtices do percurso)}	
			\IF{$vb$ melhor que $v$ (no sentido em que apresenta maior lucro)}	
				\STATE Trocar o $v$ por $vb$.	
				\IF{Solu��o inv�lida}	
					\STATE Repor $v$.	
				\ELSE
					\STATE Terminar procura e passar ao v�rtice seguinte.
				\ENDIF	
			\ENDIF	
		\ENDFOR	
	\ENDFOR	
\ENDWHILE
\end{algorithmic}
\end{algorithm}

%\section{Conclus�es}

\appendix
%\section{C�digo do Programa}

\end{document}
