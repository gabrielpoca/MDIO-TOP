\documentclass[a4paper]{article}
\usepackage[portuges]{babel}
\usepackage[latin1]{inputenc}
\usepackage[T1]{fontenc}
\usepackage{fancyvrb}
\usepackage{url}

\usepackage{aeguill}  % usefull for pdflatex
%\usepackage[compat2,twosideshift=0mm,left=20mm,right=20mm,bottom=20mm,top=15mm]{geometry}

\usepackage{algorithmic}

\parindent=2em


\title{Projecto (Integrado) de xxx\\ Relat�rio da Fase xx\\ Grupo yyyy}
\author{Gabriel Po�a (56974) \and Autor2 (n�mero) \and Autor3 (n�mero)}
\date{\today}

\begin{document}

\maketitle

\begin{abstract}

\end{abstract}

\parskip=0mm
%\tableofcontents
\parskip=2mm

\section{Exercicio 3}
Pseudo c�digo.

\begin{algorithmic}
\STATE $graph =$ Array com todos os v�rtices. 
\STATE $dMax =$ Valor da dist�ncia m�xima que cada veiculo pode percorrer.
\STATE $paths\gets$ Caminhos mais curtos a partir de $graph$ e $dMax$.
\WHILE{Existir vertices n�o visitados por analisar}
	\STATE $bestV \gets$ V�rtice de maior lucro n�o visitado.	
	\FOR{$path : paths$ (para todos os percursos)}	
		\FOR{$v : path$ (para todos os v�rtices do percurso)}	
			\IF{$vb$ melhor que $v$ (no sentido em que apresenta maior lucro)}	
				\STATE Trocar o $v$ por $vb$.	
				\IF{Solu��o inv�lida}	
					\STATE Repor $v$.	
				\ENDIF	
			\ENDIF	
		\ENDFOR	
	\ENDFOR	
\ENDWHILE

\end{algorithmic}

\section{Conclus�es}

\appendix
\section{C�digo do Programa}

\end{document}
